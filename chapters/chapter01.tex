\section{前言以及所有}

天津大学作为一所985高校居然没有自己的模板,真的有点离谱。


我在写论文的时候出于个人兴趣写了这个模板出来,建议使用overleaf直接上传使用,如果本地编译需要注意使用biblatex来实现gb7714的引用文献格式。如何编译具体可以参见
\url{https://github.com/hushidong/biblatex-gb7714-2015}


我只能说尽量维持了模板的简洁,保证了实用性,但是本人也是\LaTeX 萌新,很多东西不是很熟悉,在使用之前建议看看ctex手册。公式可以通过\ref{beginer}的方式来插入

\begin{equation}
    \bm{E} = mc^2
    \label{beginer}
\end{equation}


图像可以通过\ref{logo}的方式插入

\begin{figure}[htbp]
\centerline{\includegraphics[width=0.4\textwidth]{logo.jpg}}
\caption{天津大学的logo}
\label{logo}
\end{figure}

当然可以插入两张图片

\begin{figure}[htbp]
\centering
\begin{subfigure}{.45\textwidth}
  \centering
  \includegraphics[width=\linewidth] {logo.jpg}  
  \caption{天津大学的logo}
  \label{celoss}
\end{subfigure}
\begin{subfigure}{.45\textwidth}
  \centering
  \includegraphics[width=\linewidth]{logo.jpg}  
  \caption{天津大学的logo}
  \label{mseloss}
\end{subfigure}
\caption{俩logo}
\label{pretextloss}
\end{figure}

三线表建议使用在线表格绘制工具 \url{https://www.tablesgenerator.com/},当然要记得自己加上控制语句来控制是否需要居中等,这里也非常鼓励大家使用$tabularx$类库来控制表格,可以控制出各种长度的表格。


\begin{table}[htbp]
\centering
\begin{tabular}{cc}
\hline
食堂编号 & 食堂名称 \\ \hline
学一   & 梅园   \\
学二   & 兰园   \\
学四   & 竹园   \\ \hline
\end{tabular}
\end{table}


由于自己在写论文的时候遇到的奇怪的问题,需要调整的格式比较少,所以本身也没有做很多很复杂的功能,需要遇到问题多多百度,保持自己的求知欲。

最后,预祝大家顺利毕业。